\phantomsection
\subsection*{4A. Munich Helles}
\addcontentsline{toc}{subsection}{4A. Munich Helles}

\textbf{Impressão Geral}:Uma lager alemã de cor dourada, com sabor maltado suave e final macio e seco. A presença sutil de lúpulo condimentado, floral e/ou herbal, aliada a um amargor contido, ajuda a manter o equilíbrio maltado — mas não doce —, tornando essa cerveja uma bebida refrescante para o dia a dia.

\textbf{Aroma}: Aroma de malte semelhante a cereais adocicados. Aroma de lúpulo condimentado, floral e/ou herbal, de intensidade baixa a moderadamente baixa. Perfil de fermentação agradável e limpo, com o malte dominando o equilíbrio. Exemplares mais frescos tendem a apresentar um aroma de malte ainda mais adocicado.

\textbf{Aparência}: Cor de amarelo-claro a dourado-claro. Aparência límpida. Colarinho branco, cremoso e persistente.

\textbf{Sabor}:Início moderadamente maltado, com uma sugestão de dulçor. Sabor de malte adocicado, semelhante a cereais, de intensidade moderada, com uma impressão macia e arredondada, sustentada por um amargor de baixo a médio-baixo. Final macio e seco, sem ser bem definido ou cortante. Sabor de lúpulo condimentado, floral e/ou herbal, de intensidade baixa a moderadamente baixa. O malte domina o lúpulo no sabor, no final e no retrogosto, mas o lúpulo deve ser perceptível. Sem dulçor residual — apenas a impressão maltada com amargor contido. Perfil de fermentação limpo.

\textbf{Sensação na Boca}: Corpo médio. Carbonatação média. Perfil suave, resultado de uma maturação a frio bem conduzida (lagering).

\textbf{Comentários}: Exemplares muito frescos podem apresentar um caráter de malte e lúpulo mais proeminente, que tende a se dissipar com o tempo — algo frequentemente observado em cervejas exportadas. A Helles servida em Munique tende a ser uma versão mais leve do que aquelas produzidas fora da cidade. Também pode ser chamada de Helles Lagerbier.

\textbf{História}: Criada em Munique em 1894 para competir com as cervejas claras do tipo Pilsner, sendo geralmente creditada pela primeira vez à cervejaria Spaten. É mais popular no sul da Alemanha.

\textbf{Ingredientes}: Malte Pilsner da Europa continental, lúpulos tradicionais alemães e levedura lager alemã limpa.

\textbf{Comparação de Estilos}: Equilíbrio e amargor similares aos da Munich Dunkel, porém com uma natureza menos adocicada do malte e cor clara, ao invés de escura e rica. Apresenta maior corpo e presença de malte do que uma German Pils, mas com definição menos acentuada e menor caráter de lúpulo. O perfil de malte é semelhante ao de uma German Helles Exportbier, porém com menos lúpulo no equilíbrio e teor alcoólico levemente inferior. Possui menos corpo e álcool do que uma Festbier.

\begin{tabular}{@{}p{35mm}p{35mm}@{}}
  \textbf{Estatísticas}: & OG: 1,044 - 1,048 \\
  IBU: 16 - 22  & FG: 1,006 - 1,012  \\
  SRM: 3 - 5   & ABV: 4,7\% - 5,4\%
\end{tabular}

\textbf{Exemplos Comerciais}: Augustiner Lagerbier Hell, Hacker-Pschorr Münchner Gold, Löwenbraü Original, Paulaner Münchner Lager, Schönramer Hell, Spaten Münchner Hell, Weihenstephaner Original Helles.

\textbf{Última Revisão}: Munich Helles (2015)

\textbf{Atributos de Estilo}: bottom-fermented, central-europe, lagered, malty, pale-color, pale-lager-family, standard-strength, traditional-style
